\documentclass[12pt]{article}
\usepackage[brazil]{babel}
\usepackage[utf8]{inputenc}
%\ usepackage[printwatermark]{xwatermark}
%\ usepackage{xcolor}
\usepackage[pdftex]{graphicx}
\usepackage{hyperref}

\usepackage{color}
\definecolor{bg}{rgb}{0.93,0.93,0.9}
%\definecolor{bg}{rgb}{0.1,0.1,0.0}

% colorir código fonte
\usepackage{minted}
\usemintedstyle{tango}

\usepackage{fancyhdr}

\usepackage{draftwatermark}
\SetWatermarkScale{4}
\SetWatermarkText{Rascunho}

\usepackage{framed, color}
\definecolor{shadecolor}{rgb}{0.93,0.93,0.9}

% \newwatermark*[allpages,color=bg,angle=45,scale=3,xpos=0,ypos=0]{DRAFT}

\pagestyle{fancy}
\fancyhf{}
\fancyhead[LE,RO]{Share\LaTeX}
\fancyhead[RE,LO]{\textbf{Red} Programming Language}
\fancyfoot[CE,CO]{\leftmark}
\fancyfoot[LE,RO]{\thepage}

\title{\huge \textbf{\textcolor{red}{Red}} \\Programming Language}
\author{Guaracy Monteiro \\ guaracy.bm@gmail.com} %\thanks{emcs, \LaTeX , \AucTeX}
\date{Outubro 2015}

\begin{document}
\begin{titlepage}
\begin{figure}  %[h!t]
\centering
\includegraphics[scale=1]{Red_Language_Tower_Logo.png}
\end{figure}
\maketitle
\end{titlepage}

\tableofcontents
\pagebreak
\renewcommand\listingscaption{Listagem}
\renewcommand\listoflistingscaption{Listagens}
\listoflistings
\pagebreak


\section{Introdução}

\textit{\textcolor{red}{\textbf{ATENÇÃO:} A linguagem encontra-se em desenvolvimento e não está apta para ser
usada em produção. Muita coisa poderá ser alterada neste documento.}}

\textit{\textcolor{red}{\textbf{OBS:} Para salientar a sintaxe dos programas em Red no \LaTeX, estou usando o
\textbf{``minted''} que usa o \textbf{``pygmentize''} para efetuar a tarefa.
Para ficar correto, é necessário que seja colocado no início \textbf{Red [ ]}.}}

\subsection{Objetivo}

O objetivo inicial não é o de ser um manual, livro ou algo do gênero sobre a
linguagem. Apenas um local para que eu possa agrupar as informações e o
conhecimento sobre a linguagem. Em segundo lugar, compartilhar a linguagem com
quem estiver interessado. Vender e ficar rico está fora de cogitação. :D

\subsection{A Linguagem}

A linguagem Red é fortemente baseada em REBOL compartilhando, entre outros, a
homoiconicidade, o grande número de tipos de dados, a mistura código+data como
em Lisp. Como diferenças é possível citar a possibilidade de gerar executáveis
em código nativo (não precisa ser na mesma plataforma de desenvolviemnto) e a
tipagem opcional para parâmetros nas funções.

\subsection{Configuração}

Para criar um ambiente de desenvolvimento não são necessários poderes especiais.
A primeira coisa a fezer é baixar de 
\href{http://www.red-lang.org/p/download.html}{www.red-lang.org} a versão para o
seu sistema operacional e colocar no local que ficar mais conveniente. Note que
para o Linux, a versão disponível é de 32 bits. Para rodar em uma instalação de
64 bits é necessário instalar algumas bibliotecas para suportar a versão. As
formas mais comuns de executar o programa são:

\begin{itemize}
\item Apenas executar o programa \textbf{red} e entrará no REPL (ead–eval–print loop),
  isto é, um ambiente interativo onde você vai digitando e executando as
  instruções.
\item Executando \textbf{red \textless arquivo.red\textgreater} o script
  existente no arquivo será executado.
\item Executando \textbf{red -c \textless arquivo.red\textgreater} o script
  existente no arquivo será compilado e irá gerar um executável para a
  plataforma atual.
\item Executando \textbf{red -c -t \textless plataforma\textgreater \textless
    arquivo.red\textgreater}, o script será compilado para a plataforma
  especificada. Assim você pode estar no Linux e gerar executáveis, por exemplo,
  para Linux, Windows, Android e OSX, sem a necessidade de trocar de ambiente.
  As plataformas disponíveis são:
  \begin{itemize}
  \item \textbf{MSDOS} : Windows, x86, aplicações console (+ GUI)
  \item \textbf{Windows} : Windows, x86, GUI applications
  \item \textbf{Linux} : GNU/Linux, x86
  \item \textbf{Linux-ARM} : GNU/Linux, ARMv5, armel (soft-float)
  \item \textbf{RPi} : GNU/Linux, ARMv5, armhf (hard-float)
  \item \textbf{Darwin} : MacOSX Intel, apenas aplicações console
  \item \textbf{Syllable} : Syllable OS, x86
  \item \textbf{FreeBSD} : FreeBSD, x86
  \item \textbf{Android} : Android, ARMv5
  \item \textbf{Android-x86} : Android, x86
  \end{itemize}

\end{itemize}

\section{Sintaxe}

Antes de começar qualquer coisa, aprender um pouco da sintaxe é importante. Até
porque você deve estar acostumado com aquelas linguagens complicadas onde é
necessário separar algumas coisas com vírgula, outras com ponto e vírgula,
outras com chaves, outras com colchetes, etc., etc., etc..
Então vamos lá:

\subsection{Delimitadores}

Basicamente são três os delimitadores. Para string, blocos e caminho.
\begin{itemize}
\item \textbf{Strings} : utiliza-se aspas ( \textbf{" "} ) para strings que não
  possuam quebra de linha no interior ou chaves (\textbf{ \{ \}} ) caso a string
  tenha mais de uma linha. 
\item \textbf{Blocos} : os blocos são delimitados por colchetes ( \textbf{[ ]} )
\item \textbf{Caminhos} : são delimitados (ou concatenados) com a barra
  invertida ( \textbf{\textbackslash} 
)
\end{itemize}

\subsection{Sintaxe livre}
O delimitador padrão é o espaço e, a única restrição é separar os tokens por um
ou mais espaços. Os códigos abaixo são todos válidos e possuem a mesma
avaliação:

\begin{listing}[ht]
\begin{minted}[bgcolor=bg,
fontsize=\scriptsize,
linenos,
showtabs=true,
]{Red}
Red []
while [a > 0][print "loop" a: a - 1]

while [a > 0]
  [print "loop" a: a - 1]

while [
  a > 0
][
  print "loop" 
  a: a - 1
]

while [
  a > 0
][
  print "loop" 
  a: a - 1
]
\end{minted}
\caption{Free-form syntax}
\label{listing:1}
\end{listing}

Note que, se você entrar com \textbf{a\textless0} ou \textbf{a-1} (sem espaços) causará
um erro. Ou melhor, poderá causar um erro já que serão consideradas como
palavras (variáveis) e poderão existir e conter um valor válido. 

\subsection{Comentários}
Existem dois tipos de comentários (trechos que são ignorados pelo programa): 
\begin{itemize}
\item O comentário que inicia com ponto e vírgula ( \textbf{;} ) e vai até o
  final da linha e pode ser utilizado em qualquer parte do programa e
\item o comentário com mais de uma linha que inicia com \textbf{comment \{} e
  termina com um fecha chave ( \textbf{\}} ) pode ser utilizado em qualquer
  parte do programa menos no meio de uma expressão.
\end{itemize}

\section{REPL}
Em vez de criar um script em um editor, executar e/ou compilar, acredito que o 
mais interessante no início seja digitar e ver o resultado. Para tal, basta usar
o REPL (read-eval-print-loop). Como o nome já diz, ele lê uma entrada efetuada
pelo usuário, efetua uma avaliaçã, mostra o resultado e fica esperando uma nova
entrada. Para iniciar, basta executar \textbf{red} sem nenhum argumento e deverá
aparecer algo como:

\begin{shaded}
\begin{verbatim}
--== Red 0.5.4 ==-- 
Type HELP for starting information. 

red>>
\end{verbatim}
\end{shaded}

Digitando help e enter, serão apresentadas algumas opções para auxílio.

\section{Variáveis}
\section{Tipos de dados}
\section{Expressões}
\section{Funções}
\section{Escopo}
\section{Operadores}
\section{Controle de fluxo}
\section{Excessões}
\section{Pilha}
\section{Depuração}
\section{Estrutura do sistema}
\section{Palavras reservadas}
\section{VID}


\end{document}

