\documentclass[12pt,a4paper]{article}
%\usepackage{color}
\usepackage[dvipsnames]{xcolor}
%\usepackage{alltt}
%\usepackage[T1]{fontenc}
%\usepackage{ucs}
\usepackage[utf8x]{inputenc}

\usepackage[brazil]{babel}
%\ usepackage[printwatermark]{xwatermark}
\usepackage[pdftex]{graphicx}
\usepackage[colorlinks=true,linkcolor=blue]{hyperref}
%\usepackage{hyperref}

\definecolor{bg}{rgb}{0.16,0.16,0.16}

% colorir código fonte
\usepackage{minted}
\usemintedstyle{tango}

\usepackage{fancyhdr}

\usepackage{draftwatermark}
\SetWatermarkScale{4}
\SetWatermarkText{Rascunho}


\usepackage{framed, color}
\definecolor{shadecolor}{rgb}{0.93,0.93,0.9}

% \newwatermark*[allpages,color=bg,angle=45,scale=3,xpos=0,ypos=0]{DRAFT}

\pagestyle{fancy}
\fancyhf{}
\fancyhead[LE,RO]{Share\LaTeX}
\fancyhead[RE,LO]{\textbf{Red} Programming Language}
\fancyfoot[CE,CO]{\leftmark}
\fancyfoot[LE,RO]{\thepage}

% Style definition file generated by highlight 3.23, http://www.andre-simon.de/ 

% Highlighting theme: vim peaksea 

\newcommand{\hlstd}[1]{\textcolor[rgb]{0,0,0}{#1}}
\newcommand{\hlnum}[1]{\textcolor[rgb]{0.56,0.44,0}{#1}}
\newcommand{\hlesc}[1]{\textcolor[rgb]{0.56,0.44,0}{#1}}
\newcommand{\hlstr}[1]{\textcolor[rgb]{0,0.44,0.41}{#1}}
\newcommand{\hlpps}[1]{\textcolor[rgb]{0,0.44,0.41}{#1}}
\newcommand{\hlslc}[1]{\textcolor[rgb]{0.38,0.38,0}{#1}}
\newcommand{\hlcom}[1]{\textcolor[rgb]{0.38,0.38,0}{#1}}
\newcommand{\hlppc}[1]{\textcolor[rgb]{0,0,0}{#1}}
\newcommand{\hlopt}[1]{\textcolor[rgb]{0,0,0}{#1}}
\newcommand{\hlipl}[1]{\textcolor[rgb]{0.8,0.32,0.54}{#1}}
\newcommand{\hllin}[1]{\textcolor[rgb]{0.38,0.38,0}{#1}}
\newcommand{\hlkwa}[1]{\textcolor[rgb]{0.13,0.38,0.66}{#1}}
\newcommand{\hlkwb}[1]{\textcolor[rgb]{0.03,0.31,0.63}{#1}}
\newcommand{\hlkwc}[1]{\textcolor[rgb]{0.66,0.41,0.13}{#1}}
\newcommand{\hlkwd}[1]{\textcolor[rgb]{0.13,0.66,0.15}{#1}}
\definecolor{bgcolor}{rgb}{0.88,0.88,0.88}




\newcommand{\inchf}[2]{
  \begin{listing}[ht]
    \begin{shaded}
      \small{
        \input{./fragmentos/#1.red.tex}
      }
    \end{shaded}
    \caption{#2}
    \label{listing:3}
  \end{listing}
}

\newcommand{\incss}[1]{
  \ttfamily\small
  \begin{shaded}
    \VerbatimInput{./fragmentos/#1.ss}
  \end{shaded}
  \normalfont
}

\newcommand{\Red}{\textbf{\textcolor{red}{Red }}}

\newcommand{\codigo}[1]{\ttfamily\small\textcolor{Brown}{\textbf{#1}}\normalfont}

\title{\huge \textbf{\textcolor{red}{Red}} \\Programming Language}
\author{Guaracy Monteiro \\ \small https://github.com/guaracy/red} %\thanks{emcs, \LaTeX , \TeX}
\date{\today}

\begin{document}
\begin{titlepage}
\begin{figure}  %[h!t]
\centering
\includegraphics[scale=1]{Red_Language_Tower_Logo.png}
\end{figure}
\maketitle
\end{titlepage}

\setcounter{tocdepth}{2}
%level -1: part, 0: chapter, 1: section, etc.

\tableofcontents
\pagebreak
\renewcommand\listingscaption{Listagem}
\renewcommand\listoflistingscaption{Listagens}
\listoflistings
\listoftables
\pagebreak

\shorthandoff{"}

\section{Introdução}

\subsection{Objetivo}

O objetivo inicial não é o de ser um manual, livro ou algo do gênero sobre a
linguagem. Apenas um local para que eu possa agrupar as informações e o
conhecimento sobre a linguagem. Em segundo lugar, compartilhar a linguagem com
quem estiver interessado. Vender e ficar rico está fora de cogitação. :D

\subsection{A Linguagem}

A linguagem \Red é fortemente baseada em REBOL compartilhando, entre outros, a
homoiconicidade, o grande número de tipos de dados, a mistura código+data como
em Lisp. Como diferenças é possível citar a possibilidade de gerar executáveis
em código nativo (não precisa ser na mesma plataforma de desenvolvimento) e a
tipagem opcional para parâmetros nas funções.

\subsection{Configuração}

Para criar um ambiente de desenvolvimento não são necessários poderes especiais.
A primeira coisa a fazer é baixar de 
\href{http://www.red-lang.org/p/download.html}{www.red-lang.org} a versão para o
seu sistema operacional e colocar no local que ficar mais conveniente. Note que
para o Linux, a versão disponível é de 32 bits. Para rodar em uma instalação de
64 bits é necessário instalar algumas bibliotecas para suportar a versão. As
formas mais comuns de executar o programa são:

\begin{itemize}
\item Apenas executar o programa \textbf{red} e entrará no REPL (ead–eval–print loop),
  isto é, um ambiente interativo onde você vai digitando e executando as
  instruções.
\item Executando \textbf{red \textless arquivo.red\textgreater} o script
  existente no arquivo será executado.
\item Executando \textbf{red -c \textless arquivo.red\textgreater} o script
  existente no arquivo será compilado e irá gerar um executável para a
  plataforma atual.
\item Executando \textbf{red -c -t \textless plataforma\textgreater \textless
    arquivo.red\textgreater}, o script será compilado para a plataforma
  especificada. Assim você pode estar no Linux e gerar executáveis, por exemplo,
  para Linux, Windows, Android e OSX, sem a necessidade de trocar de ambiente.
  As plataformas disponíveis são:
  \begin{itemize}
  \item \textbf{MSDOS} : Windows, x86, aplicações console (+ GUI)
  \item \textbf{Windows} : Windows, x86, GUI applications
  \item \textbf{Linux} : GNU/Linux, x86
  \item \textbf{Linux-ARM} : GNU/Linux, ARMv5, armel (soft-float)
  \item \textbf{RPi} : GNU/Linux, ARMv5, armhf (hard-float)
  \item \textbf{Darwin} : MacOSX Intel, apenas aplicações console
  \item \textbf{Syllable} : Syllable OS, x86
  \item \textbf{FreeBSD} : FreeBSD, x86
  \item \textbf{Android} : Android, ARMv5
  \item \textbf{Android-x86} : Android, x86
  \end{itemize}

\end{itemize}

\section{Sintaxe}

Antes de começar qualquer coisa, aprender um pouco da sintaxe é importante. Até
porque você deve estar acostumado com aquelas linguagens complicadas onde é
necessário separar algumas coisas com vírgula, outras com ponto e vírgula,
outras com chaves, outras com colchetes, etc., etc., etc..
Então vamos lá:

\subsection{Delimitadores}

Basicamente são três os delimitadores. Para string, blocos e caminho.
\begin{itemize}
\item \textbf{Strings} : utiliza-se aspas ( \texttt{"string"} ) para strings que não
  possuam quebra de linha no interior ou chaves (\textbf{ \{string até\\ aqui\}} ) caso a string
  tenha mais de uma linha. 
\item \textbf{Blocos} : os blocos são delimitados por colchetes ( \textbf{[ ]} )
  e não possuem limite.
\item \textbf{Caminhos} : são delimitados (ou concatenados) com a barra
  invertida ( \textbf{\textbackslash} )
\end{itemize}

\subsection{Sintaxe livre}
O delimitador padrão é o espaço e, a única restrição é separar os tokens por um
ou mais espaços. Os códigos abaixo são todos válidos e possuem a mesma
avaliação:

\inchf{freeformsyntax}{Sintaxe livre}

Note que, se você entrar com  \textbf{a $<$ 0} ou \textbf{a-1} (sem espaços) causará
um erro. Ou melhor, poderá causar um erro já que serão consideradas como
palavras (variáveis) e poderão existir e conter um valor válido. 

\subsection{Comentários}
Existem dois tipos de comentários (trechos que são ignorados pelo programa): 
\begin{itemize}
\item O comentário que inicia com ponto e vírgula ( \textbf{;} ) e vai até o
  final da linha e pode ser utilizado em qualquer parte do programa e
\item o comentário com mais de uma linha que inicia com \textbf{comment \{} e
  termina com um fecha chave ( \textbf{\}} ) pode ser utilizado em qualquer
  parte do programa menos no meio de uma expressão.
\end{itemize}

\section{REPL}
Em vez de criar um script em um editor, executar e/ou compilar, acredito que o 
mais interessante no início seja digitar e ver o resultado. Para tal, basta usar
o REPL (read-eval-print-loop). Como o nome já diz, ele lê uma entrada efetuada
pelo usuário, efetua uma avaliação, mostra o resultado e fica esperando uma nova
entrada. Para iniciar, basta executar \textbf{red} sem nenhum argumento e deverá
aparecer algo como:

\incss{repl}

Para sair digite \textbf{q} ou \textbf{quit} e pressione enter. Digitando help e
enter, serão apresentadas algumas opções para auxílio. Lembre-se que o 

Apesar de não necessitar a digitação de \textbf{Red [ ]} que aparecem nas
listagens para efeitos de salientar a sintaxe do script, se entrar no REPL não
terá problema nenhum. O REPL entende que a entrada de uma nova linha seja a
indicação para que ele avalie a entrada. Faz-se necessário que o comando seja
digitado em uma linha a menos que ele termine com a abertura de um bloco
\textbf{[}. Neste caso, ele mudara o prompt de
\textbf{red $>>$} para uma abertura de colchetes  \textbf{[}
indicando que está esperando o fechamento para avaliar a expressão.

\incss{repl-input}

Utilizando as setas para cima e para baixo é possível navegar no histórico para a
execução de expressões informadas anteriormente. Se você digitar algo e
pressionar tab, o REPL irá mostrar uma relação das possíveis funções que podem
ser entradas, inclusive as definidas pelo usuário. Se for digitado a e tab,
teremos algo como:

\incss{repl-tab}

Se você digita \textbf{help} ou \textbf{?} seguido de uma função, será mostrado
um resumo da função informando como ela é utilizada, uma breve descrição da
função, os argumentos e alguns refinamentos. Para insert, temos:

\incss{repl-help}


\section{Variáveis}

Toda a linguagem possui alguma forma de armazenar um determinado valor em algum
lugar No caso de \Red variáveis (ou palavras) podem armazenar (ao associar)
dados ou código. 

\subsection{Nomenclatura}

\subsection{Operação}

A associação ou atribuição é feita seguindo a variável com dois pontos (
\textbf{:} ). Por exemplo, \codigo{nome: "Fulano de Tal"} irá atribuir
\codigo{"Fulano de Tal"} a variável \codigo{nome}. Para avaliar a variável e
retornar o seu valor, basta informar o nome da variável. No caso de termos 
\codigo{cliente: nome}, indica que iremos atribuir o conteúdo da variável
\codigo{nome} para a variável \codigo{cliente}. Existem casos onde não é
possível a construção acima como no caso onde o conteúdo da variável é uma
função. Para tal, precede-se o nome da variável com dois pontos e será retornado
o conteúdo mas não será avaliado. Finalmente podemos tratar a variável como
símbolo e retornará o nome da variável. Por exemplo: 

\incss{repl-var}

Podemos ver que \Red não é uma linguagem tipada, isto é, as variáveis podem
conter qualquer valor mas, depois de assumirem um valor, possuirão um tipo
correspondente que será utilizado para validar a avaliação das expressões. No
caso de funções, é possível especificar o tipo de parâmetro que será aceito.
Nada impede que a função trate os diversos tipos para retornar resultados
válidos. No exemplo, \codigo{sum a + b} retornou um erro pois não foi possível
adicionar um inteiro em um bloco (no caso pode ser considerado como uma lista).
Bastiria que a função, por exemplo, adicionasse o número em cada um dos
elementos. Da mesma forma, para atribuir a função \codigo{sum} para a variável
\codigo{x} foi necessário utilizar o formato \codigo{x: :sum} para que a função
não fosse avaliada. A construção \codigo{x: sum 2 3} seria validada e retornaria
o valor \codigo{5}. Portanto:

\begin{center}
\begin{tabular}{|l|l|}
\hline
\textbf{Formato} & \textbf{Significado} \\
\hline
\hline
\codigo{var} & Avalia a variável e retorna o resultado. \\
\hline
\codigo{var:} & Atribui um valor a uma variável. \\
\hline
\codigo{:var} & Retorna o valor de uma variável sem avaliá-lo.\\
\hline
\codigo{'var} & Trata a variável como um simbolo e retorna o valor sem avaliá-lo.\\
\hline
\end{tabular}
\end{center}

\section{Tipos de dados}

Este é um tópico onde REBOL é muito rico e, por consequencia, \Red também. A
variedade de tipos é interessante para evitar a necessidade de criar código para
trabalhar com os diversos dados requeridos pelo programa. Por exemplo, se voce
deseja trabalhar com percentuais em determinada tarefa qual a melhor forma?
Informar o decimal correspondente (e.g. 0,1 para 10\%) e efetuar diretamente o
cáculo ou informar o percentual e efetuar o cálculo no estilo \codigo{$valor \times 
  percentual \div 100$} ? Ficaria algo assim:

\incss{repl-percent}

No momento, estão disponíveis os seguintes tipos (alguns não serão utilizados em
condições normais e, mais importante, alguns ainda não foram definidos
como \codigo{data} e \codigo{hora}. Por enquanto temos:\\
\codigo{action!, binary!, bitset!, block!, char!, datatype!, error!, file!,\\
float!, function!, get-path!, get-word!, hash!, integer!, issue!, \\
lit-path!, lit-word!, logic!, map!, native!, none!, object!, op!, \\
pair!, paren!, path!, percent!, point!, refinement!, routine!, \\
set-path!, set-word!, string!, tuple!, typeset!, unset!, url!, \\
vector!, word!}. Alguns não importam para a programação normal, sendo mais
relevantes internamente.

Para você descobrir o tipo de uma variável basta entrar com \codigo{type?} e uma
variável ou um valor e será retornado o tipo atual. Para comparar com um tipo
específico, basta utilizar o tipo trocando a exclamação por uma interrogação.
Por exemplo, se quiser verificar se uma variável possui um valor inteiro, basta
digitar \codigo{integer?} variável ou valor e retornara um verdadeiro ou falso.

Para converter um tipo de dado em outro, utilizamos \codigo{to}, o tipo de dado
que desejamos e o valor a ser convertido.

\incss{repl-to}

Internamente, algumas conversões são executadas automaticamente para que um
determinado cálculo seja efetuado. Converter um valor inteiro para ponto
flutuante, por exemplo.

\subsection{binary!}

TBD

\subsection{bitset!}

TBD

\subsection{block!}

Blocos são grupos de código e/ou dados e/ou outros blocos e/ou qualquer coisa.
São delimitados por colchetes \codigo{[ ]}. Imagine, em outras linguagens, uma
matriz que pode conter qualquer coisa. Podemos efetuar operações em um bloco ou
mesmo entre blocos.

\incss{repl-block}

\subsection{char!}

Representa um caractere qualquer. O caractere é escrito entre aspas precedido de
\codigo{\#}. Para representarmos o caractere \codigo{A} usamos
\codigo{\#"A"}. Se você esquecer o \codigo{\#}, será tratado como uma
string. Para representar caracteres com valores menor que 32 (espaço),
prefixamos com um sinal circunflexo. Para o caractere 27 (escape) temos
\codigo{\#"\^{}["}. O ambiente já define alguns caracteres para
facilitar a utilização em algumas ocasiões. 

\incss{repl-char}

\subsection{datatype!}

Uso interno.

\codigo{type? error! = datatype!}

\subsection{error!}

TBD.

\subsection{file!}

TBD

\subsection{float!}

Valores numéricos não inteiros. Você pode usar o ponto ou a virgula para separar
a parte decimal. Os valores \codigo{123.45} e \codigo{123,45} são iguais. Também
é possível utilizar \codigo{e} (maiúsculo ou minúsculo) para informar um
expoente. Por exemplo \codigo{1.2345e3 = 1234.5}

\subsection{function!}

Indicam que a variável contém uma função. Você poderá utilizar \codigo{help
  função} para verificar a utilização da mesma e/ou digitar \codigo{source
  função} para ver o código fonte da função. Outras informações na seção sobre
funções (\ref{sec:func})

\incss{repl-function}

\subsection{get-path!}

TBD

\subsection{get-word!}

TBD

\subsection{hash!}

TBD

\subsection{integer!}

TBD

\subsection{issue!}

TBD

\subsection{lit-path!}

TBD

\subsection{lit-word!}

TBD

\subsection{logic!}

São variáveis cujo conteúdo se restringe a falso e verdadeiro. Diferente de
algumas linguagens que consideram 0 como falso e qualquer outro valor como
verdadeiro, em \Red isto não é possível. De qualquer forma, para melhorar o
entendimento do programa, outros valores também são definidos como falso e
verdadeiro (nada que não possa ser feito na maioria das linguagens). Por
exemplo: 

\incss{repl-logic}

\subsection{map!}

TBD

\subsection{native!}

TBD

\subsection{none}

TBD

\subsection{object!}

TBD

\subsection{op!}

Você poderia perguntar: Tipo de variável operador? Não seria função? Bem,
podemos ver um operador do tipo soma como uma função. Por exemplo, \codigo{a +
  b} poderia ser visto como algo do  tipo \codigo{add a b} (Se você conheçe
Lisp, a expressão \codigo{(+ a b)} faz todo o sentido). Uma lista dos operadores
atualmente definidos: 

\incss{repl-op}

Por exemplo, podemos ter um programa onde precisamos utilizar frequentemente se
um valor pertence (∈) a um determinado 
conjunto. Podemos facilmente criar uma
função para a tarefa. Mas para a leitura poderia ficar mais fácil termos um
operador e escrever \codigo{v ∈ l} do que uma função do tipo \codigo{pertence v
  l}. Você pode trocar o ∈ por um símbolo de acesso mais fácil pelo teclado como
o Euro (Alt gr + e). Ficaria algo como:

\incss{repl-makeop}

Definimos uma função \codigo{pertence} e criamos o operador \textbf{∈}. 

\subsection{pair!}


É uma estrutura para armazenar pares ordenados. Atualmente apenas inteiros. Para
criar um par com os valores 4 para x e 5 para y entramos com: \codigo{pt: 4x5}.
Para acessar o x podemos digitar \codigo{pt/x} ou \codigo{pt/1}. Para o y
podemos digitar \codigo{pt/y} ou \codigo{pt/1}. Assim poderemos alterar os
valores individualmente. Para colocar 6 no x basta entrar \codigo{pt/x:6}.

Também são permitidas operações sobre os pares. Um número afetar ambos os
valores \codigo{2x5 + 2 $\rightarrow$ 4x7}. Se o operedor for um par,
irá afetar os valores individualmente \codigo{2x5 + 2x1 $\rightarrow$ 4x6}.

\subsection{paren!}

TBD

\subsection{path!}

TBD

\subsection{percent!}

Uma forma fácil de trabalhar com percentuais. Para que um valor seja entendido
como percentual, basta colocar o percentual como sufixo. O valor é dividido por
100 interna e automaticamente. Assim, para calcular o percentual de um
determinado valor, basta multiplicá-lo pelo percentual \codigo{120 * 10\%
 $\rightarrow$ 12} ou se desejamos saber de qual valor é um percentual, basta
dividí-lo \codigo{12 / 10\% $\rightarrow$ 120}.

Imediatamente pensamos que, para adicionar um percentual a um valor basta entrar
com \codigo{120 + 10\% $\rightarrow$ 132}. Sinto informar que, pelo menos no
atual estágio de desenvolvimento, \textbf{não}. Mas calma, nem tudo está
perdido. Colocando o tico e o teco para funcionar chegamos a uma solução, no
mínimo, elegante. Podemos redefinir o operador \codigo{+} da seguinte forma:

\incss{repl-addpercent}

Pronto. É uma das formas de tratar com o problema. Da mesma forma, poderiamos
dar um desconto (diminuir um percentual do valor) redefinindo o operador
\codigo{-}. 

\subsection{point!}

TBD

\subsection{refinment!}

TBD

\subsection{routine!}

TBD

\subsection{set-path!}

TBD

\subsection{set-word!}

TBD

\subsection{string!}

TBD

\subsection{tuple!}

Um grupo de três ou quatro valores entre 0 e 255 separados por ponto. Podem
designar cores no formato \codigo{r.g.b} ou endereços de IP, por exemplo. Se
voce entrar no REPL com \codigo{? tuple!} terá o retorno das variáveis do tipo
(inicialmente algumas cores definidas para facilitar).

\incss{repl-tuple}

\subsection{typeset!}

TBD

\subsection{unset!}

TBD

\subsection{url!}

TBD

\subsection{vector!}

TBD

\subsection{word!}

TBD


\section{Expressões}
\section{Funções}\label{sec:func}
\section{Escopo}
\section{Operadores}
\section{Controle de fluxo}
\section{Exceções}
\section{Pilha}
\section{Depuração}
\section{Estrutura do sistema}
\section{Palavras reservadas}
\section{VID}


\end{document}

